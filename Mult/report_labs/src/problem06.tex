\CWHeader{Лабораторная работа \textnumero 6. Знакомство с обработкой аудиоданных}

\CWProblem{

\textbf{Цели:} Научиться выполнять многоканальную запись аудио информации, подключать аудио обработки
реального времени, познакомится с основными механизмами обработки аудиоданных и создания
специальных эффектов.

\textbf{Задание:}
\begin{enumerate}
\item Создать проект в программе Ableton Live;
\item Добавить в проект моно аудио-дорожку. Записать на дорожку произвольный фрагмент, например вокальную
партию или текстовое сопровождение. Фрагмент должен иметь ярко выраженные перепады уровня сигнала,
а также участки с практически полным его отсутствием;
\item Добавить в проект еще одну моно аудио-дорожку и записать на нее какой-либо шумовой сигнал. Выставить
уровень громкости дорожки таким образом, чтобы шум не заглушал полезный сигнал;
\item Добавить в проект стерео аудио дорожку и записать на нее микс аудио дорожек, созданных на предыдущих
этапах. Отключить смикшированные дорожки. Сделать экспорт проекта в wave-файл.
\item Произвести выравнивание звучания дорожки микса (выравнивание по амплитуде при помощи динамических
обработок, выполнить коррекцию частотной характеристики, сделать плавные нарастания и спады и т.д.);
\item Подключить к дорожке микса подавитель шумов (gate) и добиться подавления шума в местах отсутствия
полезного сигнала; Подключить на шины посыла/возврата (return) модуляционные и пространственные эффекты (хорус,
задержка, реверберация и т. д.)
\item Отправить с дорожки микса часть сигнала на дорожки посыла/возврата так, чтобы придать ей объемное
звучание. Внимание! Использование шин посыла/возврата в пунктах 7 и 8 является необязательным. Разрешается
размешать эффекты на обрабатываемой дорожке. Однако использование указанной техники будет отмечено
дополнительным бонусным баллом. Если шины будут использованы в проекте, то в качестве результата
работы в дополнение к MP3-файлу и исходному MIDI-файлу нужно будет прислать папку проекта. Для
уменьшения объема из нее могут быть удалены файлы сэмплов, но наличие фала проекта обязательно.
\item Сделать экспорт проекта в wave-файл.
\item Результаты 4-го и 9-го этапов конвертировать в MP3-файл при помощи программы Audacity.

\end{enumerate}

\textbf{ПО:} Ableton Live, Audacity

}

\pagebreak