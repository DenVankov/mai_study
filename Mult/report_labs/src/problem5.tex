\CWHeader{Лабораторная работа \textnumero 5. Знакомство с виртуальной студией}

\CWProblem{

\textbf{Цели:} Научиться использовать виртуальные синтезаторы для проигрывания MIDI-данных, осуществлять
сведение композиции.

\textbf{Задание:}
\begin{enumerate}
\item Создать проект в программе Ableton Live;
\item Подключить к проекту к произвольной MIDI-дорожке виртуальный синтезатор Native Instruments Kontakt
Player и загрузить в него один набор ударных инструментов и один или несколько мелодических
инструментов;
\item Импортировать в проект произвольный MIDI-файл в форме треков (по-умолчанию импортируется в виде
клипов). В качестве исходного файла можно взять любой из архива MIDI.ZIP, предлагающегося к заданию.
Также можно взять свой собственный файл или наиграть несколько партий на MIDI-клавиатуре;
\item Настроить импортированные дорожки таким образом, что они выводили команды на мелодические
инструменты виртуального инструмента Kontakt;
\item Используя глушение и солирование треков определить какая из дорожек является партией барабанов и
переключить ее на набор ударных инструментов;
\item Подключить на шины посыла/возврата (return) модуляционные и пространственные эффекты (хорус,
задержка, реверберация и т. д.)
\item Отправить с дорожек часть сигнала на дорожки посыла/возврата так, чтобы придать им объемное звучаниe. Внимание! Использование шин посыла/возврата в пунктах 6 и 7 является необязательным. Разрешается
размешать эффекты на обрабатываемой дорожке. Однако использование указанной техники будет отмечено
дополнительным бонусным баллом. Если шины будут использованы в проекте, то в качестве результата
работы в дополнение к MP3-файлу нужно будет прислать папку проекта. Для уменьшения объема из нее
могут быть удалены файлы сэмплов, но наличие фала проекта обязательно.
\item Сделать экспорт проекта в wave-файл.
\item Результаты предыдущего этапа конвертировать в MP3-файл при помощи программы Audacity. Параметры
сжатия выбрать так, чтобы суммарный объем не превышал 20 Мб.
\item Отправить полученные результаты по электронной почте на адрес
oleg_kazantsev@mail.ru
\end{enumerate}

\textbf{ПО:} Ableton Live, Audacity, Native Instruments Kontakt

}

\pagebreak