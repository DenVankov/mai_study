\section{Выполнение работы}

Работа началась с поиска программы WinFact, спустя час оказалось, что в тексте лабораторной допущена ошибка, и на самом деле программа называется WinFract, она же xfractint. Жаль, что упущенный час жизни было уже не вернуть.

Изначально фрактал fern задается в файле fractint.ifs такими коэффициентами:

\includegraphics[scale=0.5]{img/08_01.png}\\

И выглядит так:

\includegraphics[scale=0.5]{img/08_02.png}\\

За левую ветвь отвечает третья строка. При построении фрактала используются четыре аффинных преобразования: поворот, масштабирование, скос и сдвиг. Рассмотрим первые два.

Поворот выполняется с помощью матрицы
$$
\begin{pmatrix}
\cos(\alpha)& -\sin(\alpha)\\
\sin(\alpha)& \cos(\alpha)& 
\end{pmatrix}
$$

Масштабирование с помощью матрицы
$$
\begin{pmatrix}
d_{x}& 0\\
0& d_{y}
\end{pmatrix}
$$

Можем получить матричное уравнение следующего вида
$$
\begin{pmatrix}
d_{x}& 0\\
0& d_{y}
\end{pmatrix}
\cdot
\begin{pmatrix}
\cos(\alpha)& -\sin(\alpha)\\
\sin(\alpha)& \cos(\alpha)& 
\end{pmatrix}
=
\begin{pmatrix}
0.2& -0.26\\
0.23& 0.22& 
\end{pmatrix}
$$

Решив его, получим $d_{x} = 0.307$, $d_{y} = 0.343$, $\alpha = 48.99$. Т.е. стандартный угол поворота равен $48.99$ градусов.

Необходимо получить матрицу для угла поворота $48.99 + 10 + 7 = 65.99$ градусов.

Пересчитаем матрицу и получим:
$$
\begin{pmatrix}
0.406 \cdot 0.307& -0.913 \cdot 0.343\\
0.913 \cdot 0.307& 0.406 \cdot 0.343
\end{pmatrix}
=
\begin{pmatrix}
0.124& -0.313\\
0.280& 0.139
\end{pmatrix}
$$

Новый набор коэффициентов:

\includegraphics[scale=0.5]{img/08_03.png}\\

Новое изображение:

\includegraphics[scale=0.5]{img/08_04.png}\\


\pagebreak
