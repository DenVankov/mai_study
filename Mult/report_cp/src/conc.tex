\section{Аналитика и выводы}

Я научилась работать с фрактальным генератором ландшафтов VistaPro, несмотря на то, что он достаточно старый, он смог удивить качеством изображения и широтой своих возможностей. Выстраивая ландшафт деталь за деталью, с интересом осваиваешь все новые и новые возможности программы. Также я научилась работать в программе Adobe Premiere Pro. Возможности Adobe Premiere Pro кажутся безграничными. Я овладела базовыми навыками обработки видео, такими как склейка и нарезка ролика, наложение эффектов на видео, наложение аудио дорожек и выравнивание их по звуку, обеспечение их плавного перехода. Также с помощью эффекта прозрачного фона keying, удалось вставить себя прямо в этот виртуальный мир. Из трудностей можно выделить отсутствие информации о кодеке и его алгоритме сжатия. Лучшее объяснение было найдено лишь на англоязычных ресурсах.
Полученные визуальные разности в сжатом и несжатом роликах  можно объяснить тем, что на видео очень много объектов вроде бы одного цвета, но из-за теней, солнца их оттенки разнятся, а алгоритм, сжимая, теряет эти особенности, считая их незначительными.

\pagebreak

