\section{Задание}

\begin{enumerate}

\item Создание анимированной последовательности облета виртуального мира.

Смоделировать в среде фрактального генератора реалистичных ландшафтов VistaPro (или аналогичного) ландшафт, содержащий: горы, снега, солнце или луну, реку с водопадом, озеро или море, деревья. Изменить цветовую палитру одного или нескольких элементов ландшафта для создания эффекта «чужой планеты».

Осуществить облет камерой полученного ландшафта с временной задержкой на крупном плане деревьев в течение 0.5-1 секунды. При построении пути облета обратить внимание на необходимость попадания в объектив всех перечисленных элементов ландшафта. Кроме того, при полете над водной поверхностью необходимо добиться эффекта отражения источника света в воде (т.н. «лунная дорожка»).

Произвести рендеринг облета ландшафта с разрешением не менее 640x480 пикселов продолжительностью от 100 до 200 кадров с сохранением в файл формата AVI с использованием кодека без потерь качества.

\item  Видеомонтаж в системе нелинейного монтажа видеопоследовательностей.

В среде Adobe Premiere (или аналогичной) создать видеоролик, содержащий:

    \begin{itemize}
    \item анимированные титры, в которых указываются фамилии автора ролика, название дисциплины, группа, год создания;

    \item фрагменты синтезированной в VistaPro видеопоследовательности, объединенные между собой как минимум двумя эффектами перехода.
    \end{itemize}

Крупноплановый фрагмент ролика необходимо замедлить средствами Adobe Premiere до 4-5 секунд.

Самостоятельно отснять 3-5ти секундный видеофрагмент с собственным участием (можно селфи) на фоне монотонной окраски, отличающейся от цветов персонажей, на любую доступную видеотехнику (допускается моб. телефон).

Наложить фрагмент живого видео с эффектом прозрачности фона (keying) и уменьшением размера фрагмента до 1/4 экрана – на замедленную сцену ролика с крупным планом дерева.

\item Создание звуковой дорожки и чистовой рендеринг.

Подобрать соответствующие сюжету звуковые дорожки, наложить их на видеоряд с синхронизацией звука и видео по основным событиям (сценам). Предусмотреть выравнивание дорожек по громкости таким образом, чтобы общая громкость звукового сопровождения была примерно на одном уровне, а также отсутствовали пиковые выбросы, приводящие к появлению искажений. Выполнить эквализацию для выравнивания общей частотной картины и предотвращения перегруженности сигнала в узких частотных

диапазонах. Особое внимание уделить спектру в области низких частот.

Опционально: сымитировать реальное акустическое окружение при помощи эффектов задержки и реверберации.

Экспортировать результат в файлы .AVI, используя 2 кодека: один кодек – без потерь качества, другой – с частичными потерями качества (предпочтительны кодеки, использующие методы DCT или Wavelet).

\item Оформление отчета по курсовому проекту.

В разделе «реферат» отчета описать используемое ПО, и технологию сжатия используемого кодека с потерей качества.

В разделе «вычислительная часть» в подготовленных роликах необходимо отобрать кадры, воспроизводящие сцены: начальная часть ролика (с титрами); замедленный крупный план; фрагмент быстрого движения с мелкими деталями.

Для каждого из отобранных кадров привести: содержимое кадра (т.н. «скриншот»); гистограмму яркостей пикселов кадра; изображение, содержащее линейную разность между сжатым и несжатым кадрами (рекомендуется её инвертировать и визуально усилить).

В разделе «аналитика и выводы» описать основные навыки, полученные в ходе работы, затруднения в ходе работы, и дать попытку объяснить полученные визуальные разности между роликами без потерь и с потерями качества с точки зрения специфики работы используемого метода сжатия.

\end{enumerate}


\pagebreak