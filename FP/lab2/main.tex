\documentclass[12pt]{article}

\usepackage{fullpage}
\usepackage{multicol,multirow}
\usepackage{tabularx}
\usepackage{ulem}
\usepackage[utf8]{inputenc}
\usepackage[russian]{babel}
\usepackage{amsmath}
\usepackage{amssymb}

\usepackage{titlesec}

\titleformat{\section}
  {\normalfont\Large\bfseries}{\thesection.}{0.3em}{}

\titleformat{\subsection}
  {\normalfont\large\bfseries}{\thesubsection.}{0.3em}{}

\titlespacing{\section}{0pt}{*2}{*2}
\titlespacing{\subsection}{0pt}{*1}{*1}
\titlespacing{\subsubsection}{0pt}{*0}{*0}
\usepackage{listings}
\lstloadlanguages{Lisp}
\lstset{extendedchars=false,
    breaklines=true,
    breakatwhitespace=true,
    keepspaces = true,
    tabsize=2
}
\begin{document}


\section*{Отчет по лабораторной работе №\,2
по курсу \guillemotleft  Функциональное программирование\guillemotright}
\begin{flushright}
Студентка группы М8О-307 МАИ \textit{Довженко Анастасия}, \textnumero 7 по списку \\
\makebox[7cm]{Контакты: {\tt tutkarma@gmail.com} \hfill} \\
\makebox[7cm]{Работа выполнена: 17.03.2019 \hfill} \\
\ \\
Преподаватель: Иванов Дмитрий Анатольевич, доц. каф. 806 \\
\makebox[7cm]{Отчет сдан: \hfill} \\
\makebox[7cm]{Итоговая оценка: \hfill} \\
\makebox[7cm]{Подпись преподавателя: \hfill} \\

\end{flushright}

\section{Тема работы}
Простейшие функции работы со списками Common Lisp.

\section{Цель работы}
Научиться конструировать списки, находить элемент в списке, использовать схему линейной и древовидной рекурсии для обхода и реконструкции плоских списков и деревьев.

\section{Задание (вариант №2.37)}
Запрограммируйте рекурсивно на языке Коммон Лисп функционал {\tt map-set (f X)}, аргументами которого являются функция одного аргумента {\tt f} и список {\tt X}, рассматриваемый как множество. Результатом вызова должно быть множество из результатов применения {\tt f} к каждому из элементов {\tt X}. В списки, представляющие множества, нет повторений, а порядок элементов не имеет значения.

\section{Оборудование студента}
Ноутбук Asus UX310U, процессор Intel Core i7-6500U CPU 2.50GHz x 4, память: 8Gb, разрядность системы: 64.

\section{Программное обеспечение}
ОС Ubuntu 16.04 LTS, компилятор clisp, текстовый редактор Sublime Text 3.

\section{Идея, метод, алгоритм}
Идея в том, чтобы рекурсивно обработать каждый элемент списка, применив к нему заданную функцию. При этом надо проверять, что этот элемент не встречался ранее.

Выполнение {\tt map-set} происходит так: получаем первый элемент списка и применяем к нему переданную функцию, делаем рекурсивный вызов с частью списка, которая следует за первым элементом. При выходе из рекурсии соединяем обработанный первый элемент с результирующим списком, при этом проверяя, что такого элемента еще нет в списке. Если передаваемый в качестве аргумента список стал пустым, рекурсивные вызовы заканчиваются.

\section{Сценарий выполнения работы}

\section{Распечатка программы и её результаты}

\subsection{Исходный код}
\lstinputlisting{./main.lisp}

\subsection{Результаты работы}
\begin{lstlisting}
(1 3 2) 
(20 30 40) 
NIL 
(1) 
(5 4 3 2) 
(1 4 9 16 25 36 49)
(11 33 22)
\end{lstlisting}

\section{Дневник отладки}
\begin{tabular}{|p{50pt}|p{130pt}|p{130pt}|p{70pt}|}
\hline
Дата & Событие & Действие по исправлению & Примечание\\
\hline
\end{tabular}

\section{Замечания автора по существу работы}
Работа показалась мне интересной, иногда я использую функции высших порядков в других языках, но никогда не задумывалась, как они работают <<под капотом>>.

\section{Выводы}
Я написала свою реализацию встроенного функционала {\tt mapcar}. В отличии от нативного функционала, моя реализация убирает дубликаты из результирующего списка. Также узнала о таких функциях обработки списков, как {\tt car, cdr, cons}.
Стоит заметить, что {\tt car} мог бы быть заменен на {\tt first}, {\tt cdr} на {\tt rest}, а {\tt cons} на {\tt list*}.

Асимптотическая сложность работы в худшем случае -- $O(n^{2})$, где $n$ -- количество элементов списка. Она достигается за счет того, что на каждом шаге мы проверяем, не дублируется ли элемент. 
\end{document}