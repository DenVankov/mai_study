\documentclass[12pt]{article}

\usepackage{fullpage}
\usepackage{multicol,multirow}
\usepackage{tabularx}
\usepackage{ulem}
\usepackage[utf8]{inputenc}
\usepackage[russian]{babel}
\usepackage{amsmath}
\usepackage{amssymb}

\usepackage{titlesec}

\titleformat{\section}
  {\normalfont\Large\bfseries}{\thesection.}{0.3em}{}

\titleformat{\subsection}
  {\normalfont\large\bfseries}{\thesubsection.}{0.3em}{}

\titlespacing{\section}{0pt}{*2}{*2}
\titlespacing{\subsection}{0pt}{*1}{*1}
\titlespacing{\subsubsection}{0pt}{*0}{*0}
\usepackage{listings}
\lstloadlanguages{Lisp}
\lstset{extendedchars=false,
    breaklines=true,
    breakatwhitespace=true,
    keepspaces = true,
    tabsize=2
}
\begin{document}


\section*{Отчет по лабораторной работе №\,5
по курсу \guillemotleft  Функциональное программирование\guillemotright}
\begin{flushright}
Студентка группы М8О-307 МАИ \textit{Довженко Анастасия}, \textnumero 7 по списку \\
\makebox[7cm]{Контакты: {\tt tutkarma@gmail.com} \hfill} \\
\makebox[7cm]{Работа выполнена: 11.04.2019 \hfill} \\
\ \\
Преподаватель: Иванов Дмитрий Анатольевич, доц. каф. 806 \\
\makebox[7cm]{Отчет сдан: \hfill} \\
\makebox[7cm]{Итоговая оценка: \hfill} \\
\makebox[7cm]{Подпись преподавателя: \hfill} \\

\end{flushright}

\section{Тема работы}
Обобщённые функции, методы и классы объектов.

\section{Цель работы}
Научиться определять простейшие классы, порождать экземпляры классов, считывать и изменять значения слотов, научиться определять обобщённые функции и методы.

\section{Задание (вариант №5.45)}
Определите обычную функцию с двумя параметрами:\\
$p$ - многочлен, т.е. экземпляр класса polynom,\\
$a$ - список действительных чисел $(a_{1} \ldots a_{n})$, где $n$ - степень многочлена $p$.

Функция должна возвращать список действительных чисел
$(d_{0} \ldots d_{n})$, таких что:
$$ P(x) = d_{0} + d_{1} \cdot (x - a_{1}) + d_{2} \cdot (x - a_{1}) \cdot (x - a_{2}) + \ldots + d_{n} \cdot (x - a_{1}) \cdot \ldots \cdot (x - a_{n}) $$


\section{Оборудование студента}
Ноутбук Asus UX310U, процессор Intel Core i7-6500U CPU 2.50GHz x 4, память: 8Gb, разрядность системы: 64.

\section{Программное обеспечение}
ОС Ubuntu 16.04 LTS, компилятор clisp, текстовый редактор Sublime Text 3.

\section{Идея, метод, алгоритм}
\textbf{Дисклеймер:} на этапе написания отчета я поняла, что, возможно, неправильно интерпретировала задачу. В первую очередь это связано с тем, что в задании отсутствует пример использования функции, которую необходимо реализовать. В моем понимании задача состоит в следующем: привести полином к виду, описанном в задании, список $a$ при этом является коэффициентами полинома.

Сначала я вывела формулу для получения $i$-ого $d$:
$$ d_{i} = a_{i} + \sum_{j=i+1}^n (-1)^{j - i + 1} \cdot d_{j}\cdot S(i, j - 1), i = \overline{n,0},$$
где $S(j, j - i)$ -- операция суммирования всех сочетаний $C_{j}^{j - i}$ элементов списка $(a_{1} \ldots a_{j})$.

Дальше запрограммировала эти вычисления. Кратко опишу работу программы:\\
Основная функция get-d принимает на вход полином и список его коэффициентов. Коэффициенты получены с помощью функции list-coefs, которая рекурсивно обрабатывает термы. Т.к. список термов разрежен, необходимо отслеживать изменения степеней, что делается во вспомогательной функции cur-coef. В итоге имеем полный список всех коэффициентов.\par
В цикле получаем $d_{i}$. Чтобы получить этот элемент, находится сумма $a_{i}$ и $\sum_{j=i+1}^n (-1)^{j - i + 1} \cdot d_{j}\cdot S(i, j - 1)$. Последнее слагаемое считается с помощью функции sum-mult-d-S, в которой считается сумма списка слагаемых. Этот список получен с помощью функции list-d-s, а сами слагаемые вычисляются в функции d-mult-S. 

\section{Сценарий выполнения работы}

\section{Распечатка программы и её результаты}

\subsection{Исходный код}
\lstinputlisting{./main.lisp}

\subsection{Результаты работы}
(58.339996 44.8 5) \\
(13 6 3 1) \\
(14724 -2454 200 -52 8 -2) \\

\section{Дневник отладки}
\begin{tabular}{|p{50pt}|p{130pt}|p{130pt}|p{70pt}|}
\hline
Дата & Событие & Действие по исправлению & Примечание\\
\hline
\end{tabular}

\section{Замечания автора по существу работы}
Для меня осталось загадкой, зачем передавать в функцию и экземпляр класса полинома, и его коэффициенты. Ведь нам достаточно иметь только список коэффициентов, чтобы получить список чисел d. Его можно получить из экземпляра класса, но если коэффициенты уже переданы как аргумент, то зачем получить их из полинома?

\section{Выводы}
Я научилась работать с простейшими классами, порождать экземпляры классов, производить различные действия над ними. Также мне пригодились навыки работы со списками, полученные в прошлых лабораторных.

\end{document}