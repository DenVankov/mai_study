\documentclass[12pt]{article}

\usepackage{fullpage}
\usepackage{multicol,multirow}
\usepackage{tabularx}
\usepackage{ulem}
\usepackage[utf8]{inputenc}
\usepackage[russian]{babel}
\usepackage{amsmath}
\usepackage{amssymb}

\usepackage{titlesec}

\titleformat{\section}
  {\normalfont\Large\bfseries}{\thesection.}{0.3em}{}

\titleformat{\subsection}
  {\normalfont\large\bfseries}{\thesubsection.}{0.3em}{}

\titlespacing{\section}{0pt}{*2}{*2}
\titlespacing{\subsection}{0pt}{*1}{*1}
\titlespacing{\subsubsection}{0pt}{*0}{*0}
\usepackage{listings}
\lstloadlanguages{Lisp}
\lstset{extendedchars=false,
    breaklines=true,
    breakatwhitespace=true,
    keepspaces = true,
    tabsize=2
}
\begin{document}


\section*{Отчет по лабораторной работе №\,4
по курсу \guillemotleft  Функциональное программирование\guillemotright}
\begin{flushright}
Студентка группы М8О-307 МАИ \textit{Довженко Анастасия}, \textnumero 7 по списку \\
\makebox[7cm]{Контакты: {\tt tutkarma@gmail.com} \hfill} \\
\makebox[7cm]{Работа выполнена: 28.03.2019 \hfill} \\
\ \\
Преподаватель: Иванов Дмитрий Анатольевич, доц. каф. 806 \\
\makebox[7cm]{Отчет сдан: \hfill} \\
\makebox[7cm]{Итоговая оценка: \hfill} \\
\makebox[7cm]{Подпись преподавателя: \hfill} \\

\end{flushright}

\section{Тема работы}
Знаки и строки.

\section{Цель работы}
Научиться работать с литерами (знаками) и строками при помощи функций обработки строк и общих функций работы с последовательностями.

\section{Задание (вариант №4.33)}
Запрограммировать на языке Коммон Лисп функцию, принимающую один аргумент - строку предложения.

Функция должна
\begin{enumerate}
\item удалить все слова с нечётными порядковыми номерами,
\item перевернуть все слова с чётными номерами,
\item вернуть новое предложение.
\end{enumerate}

\section{Оборудование студента}
Ноутбук Asus UX310U, процессор Intel Core i7-6500U CPU 2.50GHz x 4, память: 8Gb, разрядность системы: 64.

\section{Программное обеспечение}
ОС Ubuntu 16.04 LTS, компилятор clisp, текстовый редактор Sublime Text 3.

\section{Идея, метод, алгоритм}
Идея в том, чтобы разбить строку на список слов, удалить нечетные элементы из списка слов, реверсировать каждый элемент списка и объединить список в строку.

\section{Сценарий выполнения работы}

\section{Распечатка программы и её результаты}

\subsection{Исходный код}
\lstinputlisting{./main.lisp}

\subsection{Результаты работы}
"отч от олатс"\ \\
"в в ,йымюргу то ламердаз дан атнаилоф"\ \\ 
"llit eid"\ \\
"ёще хикгям ,колуб йепыв"\

\section{Дневник отладки}
\begin{tabular}{|p{50pt}|p{130pt}|p{130pt}|p{70pt}|}
\hline
Дата & Событие & Действие по исправлению & Примечание\\
\hline
\end{tabular}

\section{Замечания автора по существу работы}
Строки в Коммон Лиспе является подтипом vector, поэтому к ним применимы все функции работы с векторами и последовательностями, что очень удобно.

\section{Выводы}
Я научилась работать со строками при помощи функций обработки строк и общих функций работы с последовательностями. Также при выполнении работы мне помогли знания, полученные в предыдущих работах.

\end{document}